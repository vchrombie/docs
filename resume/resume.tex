%%%%%%%%%%%%%%%%%%%%%%%%%%%%%%%%%
%%% CV Bruno Alves 06/2017 %%%%%%
%%%%%%%%%%%%%%%%%%%%%%%%%%%%%%%%%

\PassOptionsToPackage{dvipsnames}{xcolor}
\documentclass[10pt,a4paper]{style}

%Layout
\geometry{left=1cm,right=9cm,marginparwidth=6.8cm,marginparsep=1.2cm,top=1.25cm,bottom=1.25cm,footskip=2\baselineskip}

%Packages
\usepackage[utf8]{inputenc}
\usepackage[T1]{fontenc}
\usepackage[default]{lato}
\usepackage{hyperref}

%Colors

\definecolor{accent}{HTML}{000e17}
\definecolor{heading}{HTML}{000e17}
\definecolor{emphasis}{HTML}{696969}
\definecolor{body}{HTML}{01415f}

\colorlet{heading}{heading}
\colorlet{accent}{accent}
\colorlet{emphasis}{emphasis}
\colorlet{body}{body}

\renewcommand{\itemmarker}{{\small\textbullet}}
\renewcommand{\ratingmarker}{\faCircle}

%

\begin{document}
\name{Venu Vardhan Reddy Tekula}
\tagline{}
% \photo{2.5cm}{BrunoAlves}
\personalinfo{
  \email{vt2182@nyu.edu}
%   \phone{+91 8186866445}
%   \mailaddress{Rua do Verdelho, 218, Hab.38}
%   \location{Hyderabad, India}
  \homepage{vchrombie.github.io}
  \linkedin{linkedin.com/in/tvvr}
  \github{github.com/vchrombie}
%   \twitter{twitter.com/vchrombie}
}

%

\begin{fullwidth}
\makecvheader
\end{fullwidth}

%

%%%%%%%%%%%%%%%%%%%%%%%%%%%%%%% Experience %%%%%%%%%%%%%%%%%%%%%%%%%%%%%%%

\cvsection[sidebar]{Experience}

\cvevent{Software Developer}{BeeHyv Software Solutions Pvt. Ltd}{January 2022 - June 2022}{Hyderabad}

\begin{itemize}
    \item Created the squad for working on integrating crater, an open-source invoicing solution, into Bahmni for handling the payments.
    \item Worked as the payments-lite squad leader, and my responsibilities were managing the project, implementing new features, \& fixing bugs.
\end{itemize}

\divider

\cvevent{GrimoireLab Community Manager}{Bitergia \& CHAOSS, The Linux Foundation}{June 2021 - December 2021}{Spain/Remote Work}

\begin{itemize}
    \item Worked on fostering and growing the GrimoireLab Community.
    \item Worked closely with the WG which build, document, craft solutions and maintain different projects under the GrimoireLab toolchain.
    % \item Joined the GrimoireLab MSR Virtual Hackathon Committee Team as a Support Co-Chair for the MSR Conference 2022.
    % \item Was involved in mentoring, on-boarding, and dissemination activities around GrimoireLab like the Google Season of Docs and Outreachy.
\end{itemize}

%

\divider

\cvevent{Backend Developer}{Bitergia}{October 2020 - April 2021}{Spain/Remote Work}

\begin{itemize}
	\item Worked on creating an automatic packaging \& release workflow which generates releases notes using changelog entries and publishing the releases to GitHub.
% 		\item Worked on creating an automatic packaging \& release workflow of the GrimoireLab toolset.
% 	\item Worked on creating release-tools using Python to generate releases notes using changelog entries and publishing the releases to GitHub.
% 	\item Worked closely with the customers by providing support during issues in the deployments of Bitergia Analytics. We continuously take feedback \& feature requests and work on them.
\end{itemize}

%

% %%%%%%%%%%%%%
% %Experience
% %%%%%%%%%%%%%

% \cvevent{Electrical Engineering / IT Manager}{ADIRA –- Metal-Forming Solutions S.A.}{April 2014 -- April 2015}{Porto, Portugal}

% \begin{itemize}
%   \item Development of electrical projects for Press Brakes and Shears;
%   \item Implementation and research of new features for products;
%   \item PLC Programming;
%   \item IT management and administration;
% \end{itemize}

% \divider
% %

\cvsection{Projects}

%%%%%%%%%%%%%
%Project 1
%%%%%%%%%%%%%

\cvproject{Creating Quality models using GrimoireLab and CHAOSS metrics}
\begin{itemize}
  \item The main aim is to design an approach to shape the GrimoireLab data in a format that can easily be consumed by Prosoul and implement it on the data obtained from a few data sources like git, github, and mailing list repositories to obtain simple quality models.
\end{itemize}

\divider

\cvproject{GrimoireLab: Perceval plugin for Zulip}
\begin{itemize}
  \item Worked on creating a Perceval backend plugin for the Zulip platform.
  \item Perceval is the GrimoireLab component that collects data from different data sources and returns it as JSON documents.
\end{itemize}

\divider

\cvproject{GitLit}
\begin{itemize}
  \item Social networking site exclusively for OSS developers built with Django, GitHub API, and GraphQL.
  \item Calculates scores of profiles and projects and connects enthusiastic developers to the most interesting and relevant projects and forming new communities of like-minded and passionate developers.
\end{itemize}

\divider

\cvproject{Ecotourism@Lambasingi (Student Social Responsibility Project)}
\begin{itemize}
  \item We organized a 4-day campaign to promote cleanliness in a remote place Lambasingi, which is one of the villages in Andhra Pradesh near Visakhapatnam.
\end{itemize}

%

% %%%%%%%%%%%%%%%%%%%%%%%%%%%%%%% Certificates %%%%%%%%%%%%%%%%%%%%%%%%%%%%%%%

% \cvsection{Certificates}

% %%%%%%%%%%%%%
% %Certificate 1
% %%%%%%%%%%%%%

% \cvevent{Kaizen Office Live}{Kaizen Institute Portugal}{5hours}{Porto,Portugal}
% \divider

% %%%%%%%%%%%%%
% %Certificate 2
% %%%%%%%%%%%%%

% \cvevent{Kaizen Service}{Kaizen Institute Portugal}{16hours}{Porto,Portugal}
% \divider


\end{document}
