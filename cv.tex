%%%%%%%%%%%%%%%%%%%%%%%%%%%%%%%%%%%%%%%%%
% "ModernCV" CV and Cover Letter
% LaTeX Template
% Version 1.11 (19/6/14)
%
% This template has been downloaded from:
% http://www.LaTeXTemplates.com
%
% Original author:
% Xavier Danaux (xdanaux@gmail.com)
%
% License:
% CC BY-NC-SA 3.0 (http://creativecommons.org/licenses/by-nc-sa/3.0/)
%
% Important note:
% This template requires the moderncv.cls and .sty files to be in the same 
% directory as this .tex file. These files provide the resume style and themes 
% used for structuring the document.
%
%%%%%%%%%%%%%%%%%%%%%%%%%%%%%%%%%%%%%%%%%

%----------------------------------------------------------------------------------------
%	PACKAGES AND OTHER DOCUMENT CONFIGURATIONS
%----------------------------------------------------------------------------------------


\documentclass[11pt,a4paper,sans]{moderncv} % Font sizes: 10, 11, or 12; paper sizes: a4paper, letterpaper, a5paper, legalpaper, executivepaper or landscape; font families: sans or roman


\moderncvstyle{classic} % CV theme - options include: 'casual' (default), 'classic', 'oldstyle' and 'banking'
\moderncvcolor{blue} % CV color - options include: 'blue' (default), 'orange', 'green', 'red', 'purple', 'grey' and 'black'

\usepackage{lipsum} % Used for inserting dummy 'Lorem ipsum' text into the template

\usepackage[scale=0.85]{geometry} % Reduce document margins
\setlength{\hintscolumnwidth}{2.5cm} % Uncomment to change the width of the dates column
%\setlength{\makecvtitlenamewidth}{10cm} % For the 'classic' style, uncomment to adjust the width of the space allocated to your name

%----------------------------------------------------------------------------------------
%	NAME AND CONTACT INFORMATION SECTION
%----------------------------------------------------------------------------------------

\firstname{Venu Vardhan Reddy} % Your first name
\familyname{Tekula} % Your last name

% All information in this block is optional, comment out any lines you don't need
\title{Curriculum Vitae}
%\address{Address}{City, State Zip}
\mobile{+918186866445}
\email{venuvardhanreddytekula8@gmail.com}
\homepage{vchrombie.github.io}{} 
\extrainfo{DOB: May 28, 1998}

%\photo[70pt][0.4pt]{pictures/House} % The first bracket is the picture height, the second is the thickness of the frame around the picture (0pt for no frame)
%\quote{"A witty and playful quotation" - John Smith}


% The first argument is the url for the clickable link, the second argument is the url displayed in the template - this allows special characters to be displayed such as the tilde in this example



%----------------------------------------------------------------------------------------

\begin{document}

\makecvtitle % Print the CV title

%----------------------------------------------------------------------------------------
%	INTERESTS SECTION
%----------------------------------------------------------------------------------------

\section{Area of Interests}
\cvitem{}{Software Engineering and Evolution, Open Source, Data Analytics.}

% \renewcommand{\listitemsymbol}{-~} % Changes the symbol used for lists
% \cvlistdoubleitem{Piano}{Cricket}
% \cvlistitem{Cricket}

%----------------------------------------------------------------------------------------
%	EDUCATION SECTION
%----------------------------------------------------------------------------------------

\section{Education}

\cventry{July 2016 - July 2020}{Bachelor's of Technology}{Computer Science and Engineering}{\link[Amrita School of Engineering, Amrita Vishwa Vidyapeetham]{https://amrita.edu}}{Kollam, India. CGPA: 8.26/10.0}{}

\cventry{2014 - 2016}{Higher Secondary}{}{Narayana Junior College}{Hyderabad, India. Marks: 95.1\%}{}

\cventry{2014}{Secondary}{}{Dilsukhnagar Public School}{Hyderabad, India. GPA 9.8/10.0}{}

%----------------------------------------------------------------------------------------
%	Publications SECTION
%----------------------------------------------------------------------------------------

\section{Publications}

\cvitem{2020}
{
    Iterative Text-to-Image conversion using Recurrent Generative Adversarial Models, Vinay Varma Nadimpalli, \textbf{Venu Vardhan Reddy Tekula}, Sumanth Javvaji, Sedimbi Satya Pramod, and Jyothisha J. Nair, in \link[3rd International Conference on Interdisciplinary Research \textbf{(ICIR 2020)}]{https://icir.co.in/}.
}


%----------------------------------------------------------------------------------------
%	Achievements SECTION
%----------------------------------------------------------------------------------------

\section{Experience}

\cvitem{November 2021 - Present}
{
    Hackathon Support Co-Chair, \textbf{\link[MSR 2022]{https://conf.researchr.org/home/msr-2022/}}.
    \newline
    Joined the GrimoireLab MSR Virtual Hackathon Committee Team as a Support Co-Chair for the Mining Software Repositories Conference 2022. The MSR conference is the premier conference for data science, machine learning, and artificial intelligence in software engineering.
}

\cvitem{June 2021 - Present}
{
    Community Manager, \textbf{\link[Grimoirelab]{https://grimoirelab.github.io/}, Bitergia x CHAOSS}, Remote.
    \newline
    Working on fostering and growing the GrimoireLab Community. My responsibilities include engaging the community, clarifying the doubts, and helping the new contributors. In addition, I was also involved in organizing outreach programs like the GSoD, Summer OSPP.
}

\cvitem{October 2020 - April 2021}
{
    Backend Developer (Temporary), \textbf{\link[Bitergia]{https://bitergia.com/}}, Spain/Remote Work.
    \newline
    Bitergia helps companies and organizations by understanding and improving the software development projects that matter to them. I worked on creating an automatic packaging \& release workflow of the GrimoireLab toolset using the \link[release-tools]{https://github.com/Bitergia/release-tools}, poetry, and github actions.
}

\cvitem{January 2019 - September 2020}
{
    Open Source Contributor, Student Developer (GSoC) \textbf{\link[CHAOSS]{https://chaoss.community}} - The Linux Foundation.
    \newline
    CHAOSS is an Open Source project focused on creating analytics and metrics to help and define community health. GrimoireLab is a set of free, open-source software tools for software development analytics. I work closely with the \textbf{GrimoireLab WG} that maintains the project.
    \newline
    I also worked as a Student Developer on the project "Creating Quality models using GrimoireLab and CHAOSS metrics" as a part of the \textbf{Google Summer of Code Program 2020}.
}

\cvitem{July 2016 - April 2020}
{
    Member and Student Mentor, \textbf{\link[FOSS@Amrita]{https://amfoss.in/}}, Amrita Vishwa Vidyapeetham, Amritapuri.
    \newline
    amFOSS (Open Source Club of Amrita) trains students to contribute to free and open-source software and skill development. I actively mentor juniors and participate in various club activities.
}

\cvitem{July 2019 - August 2019}
{
    Summer School Student, \textbf{\link[Ben-Gurion University of the Negev]{http://bgu.ac.il}}, Beersheba, Israel.
    \newline
    Got selected for a one-month summer course on Data Mining and Business Intelligence with its applications in Cyber Security at BGU, Israel.
}

%----------------------------------------------------------------------------------------
%	Technical Projects SECTION
%----------------------------------------------------------------------------------------

\section{Open Source Projects}

\cvitem{January 2021 - June 2021}{\textbf{release-process-automation}
\newline 
Created an automatic packaging \& release workflow of the GrimoireLab toolset using the release-tools, poetry, and github actions. Designed this workflow to manage the releases of GrimoireLab with proper versioning and release notes.
\newline 
\href{https://github.com/vchrombie/release-process-automation}{https://github.com/vchrombie/release-process-automation}}

% \cvitem{April 2021}{\textbf{gareth}
% \newline 
% Developed a tool to automate the developer installation of GrimoireLab. This tool can install the setup and can be used to update the projects from time to time to use the latest version of the toolset.
% \newline 
% \href{https://github.com/vchrombie/gareth}{https://github.com/vchrombie/gareth}}

% \cvitem{October 2020 - April 2021}{\textbf{release-tools}
% \newline 
% Contributed to the Bitergia/release-tools project which is a set of tools to generate GrimoireLab releases. With this package, GrimoireLab maintainers can automate many of the boring and time-consuming tasks related to packages and releases. Developed and tested many key functionalities of the tools.
% \newline 
% \href{https://github.com/Bitergia/release-tools}{https://github.com/Bitergia/release-tools}}

% \cvitem{August 2020}{\textbf{Ocellus}
% \newline 
% Mentored the team from amFOSS for Hac’KP, an international hackathon organized by the Kerala Police Cyberdome. This project does OSINT data analysis: IP address scans, IP address heatmapping, tracking mac addresses of a system, phone number, and email verification, and blacklist and domain analysis. It was ranked under the top 20 out of 200+ participants.
% \newline 
% \href{https://ocellus.netlify.app/}{https://ocellus.netlify.app/}}

\cvitem{May 2020 - August 2020}{\textbf{Creating Quality models using GrimoireLab and CHAOSS metrics}
\newline 
The main aim of the project is to design an approach to shape the GrimoireLab data in a format that can easily be consumed by Prosoul and implement it on the data obtained from a few data sources like git, github and mailing list repositories to obtain simple quality models.
\newline 
\href{https://github.com/vchrombie/gsoc}{https://github.com/vchrombie/gsoc}}

\cvitem{May 2020}{\textbf{GrimoireLab: Perceval backend plugin for Zulip}
\newline 
Worked on creating a perceval backend plugin for Zulip data source. Perceval is the GrimoireLab component that collects data from different data sources and returns it as JSON documents. Created a set of third-party community maintained backends for Perceval.
\newline 
\href{https://github.com/vchrombie/grimoirelab-perceval-zulip}{https://github.com/vchrombie/grimoirelab-perceval-zulip}
\newline
\href{https://github.com/perceval-backends}{https://github.com/perceval-backends}}

% \cvitem{October 2019}{\textbf{Community Hacktoberfest Tracker}
% \newline 
% I started this project during the Hacktoberfest season. A simple web app that gives the hacktoberfest stats of your whole community. Built using Flask, Bootstrap, and GitHub API.
% \newline 
% \href{https://github.com/snitch3s/hackbunch}{https://github.com/snitch3s/hackbunch}}

% \cvitem{January 2019 - April 2019}{\textbf{CHAOSS Microtasks}
% \newline 
% It was a part of the selection procedure for Google Summer of Code 2019 for the CHAOSS organization. Worked on the implementations of evolution metrics in Python and made detailed analysis reports about the growth of FOSSASIA open source projects.
% \newline 
% \href{https://github.com/vchrombie/chaoss-microtasks}{https://github.com/vchrombie/chaoss-microtasks}}

% \cvitem{January 2019 - February 2019}{\textbf{Amma Teachings} An Amazon Alexa application which gives you quotes given by \textit{Amma} whenever you trigger it with a voice call. Built using Alexa Skill Kit, Python, and Amazon AWS Lambda. The application got published in the Amazon Indian Store too. \url{https://amzn.to/2TTlvD6} \newline \href{https://github.com/vchrombie/amma-teachings}{https://github.com/vchrombie/amma-teachings}}

\cvitem{December 2018}{\textbf{GitLit}
\newline 
It is a social networking site exclusively for OSS developers. Built with Django, GitHub GraphQL, Bootstrap, and Saas. It works by connecting enthusiastic developers to the most interesting and relevant projects and forming new communities of like-minded and passionate developers.
\newline 
\href{https://gitlab.com/amfoss/gitlit}{https://gitlab.com/amfoss/gitlit}}

% \cvitem{October 2018 - December 2018}{\textbf{Grootify} A simple Chrome Extension that replaces every image of your web page with Groot. Made using HTML/CSS and JS.
% \newline \href{https://github.com/vchrombie/grootify}{https://github.com/vchrombie/grootify}}


%----------------------------------------------------------------------------------------
%	Course Projects SECTION
%----------------------------------------------------------------------------------------

\section{Academic Projects}

\cvitem{September 2019 - March 2020}{\textbf{Iterative Text-to-Image conversion using Recurrent GANs (Tell, Draw, Undo, Repeat)}
\newline
Dr. Jyothisha J. Nair, Dept. of CSE, ASE, Amritapuri.
\newline
\textit{Bachelor Thesis}
\newline 
We propose a novel Recurrent GAN model architecture that can generate 2-D images from input text in an iterative manner. This is different from the one-step text-to-image generation as the model will be given continuous instructions carrying information on how to modify the most recently generated image.
\newline
\href{http://sersc.org/journals/index.php/IJAST/article/view/22267}{http://sersc.org/journals/index.php/IJAST/article/view/22267}}

\cvitem{August 2019 - October 2019}{\textbf{Stock Market Prediction using LSTM} 
\newline 
Veena G, Dept. of CSE, ASE, Amritapuri.
\newline
\textit{15CSE481:Machine Learning and Data Mining Lab}
\newline 
This project is about analyzing and predicting the stock prices of Google using Artificial Recurrent Neural Network (RNN) architecture. This is a Regression problem.
\newline 
\href{https://github.com/vchrombie/shuri}{https://github.com/vchrombie/shuri}}

\cvitem{July 2019 - August 2019}{\textbf{Analysis of Phishing Detection using SHAP} 
\newline 
Dr. Yisroel Mirsky, Shai Cohen, Dept. of SISE, BGU, Beersheba. \newline 
I worked on the research project of BGU Cyber Labs, Phishing Detection during the Summer School program at BGU. The motive of the project to extract the important features contributing to the project and make an analysis using the SHAP tool. Used Python and Jupyter Notebooks. \newline 
\href{https://github.com/vchrombie/pd-shap-analysis}{https://github.com/vchrombie/pd-shap-analysis}}

% \cvitem{March 2019 - April 2019}{\textbf{Black Friday Sales Problem} \newline 
% Dr. Vivek Menon, Dept. of CSE, ASE, Amritapuri. 
% \newline 
% \textit{15CSE387/2:R for Data Science}
% \newline 
% This project is the team submission for the black-friday-sales Kaggle problem. Used R and RStudio for completing the challenge.
% \newline 
% \href{https://github.com/vchrombie/black-friday-sales}{https://github.com/vchrombie/black-friday-sales}}

% \cvitem{March 2019 - April 2019}{\textbf{Star Wars Analysis} This was a part of the Data Analytics Course. Made a small analysis on the star-wars dataset to find some interesting patterns.\newline \href{https://github.com/vchrombie/star-wars-analysis}{https://github.com/vchrombie/star-wars-analysis}}

\cvitem{August 2018 - November 2018}{\textbf{Malicious Web Content Detection using Machine Learning} 
\newline 
Dr. Sajeev G P, Dept. of CSE, ASE, Amritapuri. 
\newline 
\textit{15CSE265:Scientific Computing}
\newline 
A Chrome Extension checks the website you are browsing, and categorizes it into SAFE/MALICIOUS, using the Random Forests Model. Built using Python, PHP, HTML/CSS. 
\newline 
\href{https://github.com/vchrombie/cap-america}{https://github.com/vchrombie/cap-america}}

\cvitem{July 2018 - August 2018}{\textbf{EcoTourism@Lambasingi} \textit{Student Social Responsibility Project} 
\newline 
We organized a 4-day campaign to \textit{promote cleanliness} in a \textit{remote place Lambasingi}, which is one of the villages in Andhra Pradesh near Visakhapatnam as a part of the \textit{SSR Project}.}

%\cvitem{July 2017 - November 2017}{\textbf{Student Management System} \newline Dr. R. Sreekumar, Dept. of CSE, ASE, Amritapuri. \newline This is made as a part of the course 15CSE202:Object Oriented Programming. Manages student records like leave records, marks. Built using Java and Eclipse. \newline \href{https://github.com/vchrombie/java-project}{https://github.com/vchrombie/java-project}}



%----------------------------------------------------------------------------------------
%	Open Source Contributions SECTION
%----------------------------------------------------------------------------------------

\section{Open Source Contributions}
\cvitem{}{
\link[OpenSearch Dashboards]{https://github.com/opensearch-project/OpenSearch-Dashboards}, 
\link[opensearch-py]{https://github.com/opensearch-project/opensearch-py}, 
\link[Bitergia/release-tools]{https://github.com/Bitergia/release-tools}, 
\link[coveralls-python]{https://github.com/TheKevJames/coveralls-python}
}


%----------------------------------------------------------------------------------------
%	Institute Positions SECTION
%----------------------------------------------------------------------------------------

\section{Institute Positions}

\cvitem{2019}{\textbf{Head Coordinator}, Data Science Team | FOSS@Amrita.}

\cvitem{2016 - 2019}{\textbf{Student Coordinator}, Amritavarsham | MAM, Amritapuri.}

\cvitem{2018}{\textbf{Head Coordinator}, Social Media Team | FOSS@Amrita.}

\cvitem{2016 - 2019}{\textbf{Student Coordinator}, Amala Bharatham Campaign (ABC) | MAM, Amritapuri}

%----------------------------------------------------------------------------------------
%	Teaching and Mentoring SECTION
%----------------------------------------------------------------------------------------

\section{Teaching and Mentoring}

\cvitem{2021}{\textbf{Mentor}, Google Summer of Code | CHAOSS. \link[Link]{https://summerofcode.withgoogle.com/archive/2021/projects/5275752542502912/}}

\cvitem{2021}{\textbf{Mentor}, Summer of Open Source Promotion Plan | CHAOSS. \link[Link]{https://summer.iscas.ac.cn/}}

\cvitem{2019}{\textbf{Chief Coordinator}, Road To Excellence. | CIR@Amrita x FOSS@Amrita.}


%----------------------------------------------------------------------------------------
%	Talks SECTION
%----------------------------------------------------------------------------------------

\section{Talks}

\cvitem{July 2021}{\textbf{Introduction to GrimoireLab and more}
\newline
CHAOSS Beijing Meetup, Virtual Event. \link[YouTube]{https://youtu.be/dm1iamz-RKA}}


%----------------------------------------------------------------------------------------
%	Skills SECTION
%----------------------------------------------------------------------------------------

\section{Technical Skills}
%
\cvitem{Programming}{\textsc{Python, Java, JavaScript}}
\cvitem{OS}{GNU/Linux}
\cvitem{Tools}{\LaTeX, Jupyter, Git \& GitHub/GitLab, MySQL, Docker, Elasticsearch, Kibana.}
\cvitem{Frameworks}{Django, Flask, Vue.js}

%----------------------------------------------------------------------------------------
%	AWARDS SECTION
%----------------------------------------------------------------------------------------
%\section{Certifications}

%\cventry{April 2015 -- April 2017}{Certified LabVIEW Associate Developer}{\textsc{National Instruments}}{}{}{}

\section{Achievements}
%
\cvitem{InCTF, 2019}{Got selected for the finals after finishing the Qualifier round of Amrita InCTF 2019, India's First Capture the Flag Cybersecurity Contest.}
\cvitem{Summer Student, 2019}{Got selected as a student to the Data Mining and Business Intelligence with application in Cyber Security Summer Program at Ben-Gurion University of the Negev (BGU), Israel with 92\% scholarship and 400 USD stipend.}


%----------------------------------------------------------------------------------------
%	Coursework SECTION
%----------------------------------------------------------------------------------------

\section{Coursework}

%\renewcommand{\listitemsymbol}{-~} % Changes the symbol used for lists

\cvitem{Core Courses}{Database Management Systems, Functional Programming, Data Structures and Algorithms, Object-Oriented Programming, Scientific Computing, Design and Analysis of Algorithms, Software Engineering, Natural Language Processing, Information Retrieval.}

\cvitem{Lab Courses}{Database Management Systems Lab, Data Structure Lab, Object-Oriented Programming Lab, Operating System Lab, R for Data Science.}

%\cvlistdoubleitem{Basics of Electricity & Magnetism}{Computer Programming and Utilization}

%----------------------------------------------------------------------------------------
%	COMMUNICATION SKILLS SECTION
%----------------------------------------------------------------------------------------


%----------------------------------------------------------------------------------------
%	LANGUAGES SECTION
%----------------------------------------------------------------------------------------

\section{Languages}

\cvitemwithcomment{English}{Fluent}{}
\cvitemwithcomment{Hindi, Telugu}{Native}{}
% \cvitemwithcomment{Spanish}{Intermediate}{}
%\cvitemwithcomment{Dutch}{Basic}{Basic words and phrases only}

%----------------------------------------------------------------------------------------
%	COVER LETTER
%----------------------------------------------------------------------------------------

% To remove the cover letter, comment out this entire block

%\clearpage

%\recipient{HR Department}{Corporation\\123 Pleasant Lane\\12345 City, State} % Letter recipient
%\date{\today} % Letter date
%\opening{Dear Sir or Madam,} % Opening greeting
%\closing{Sincerely yours,} % Closing phrase
%\enclosure[Attached]{curriculum vit\ae{}} % List of enclosed documents

%\makelettertitle % Print letter title

%\lipsum[1-3] % Dummy text

%\makeletterclosing % Print letter signature

%----------------------------------------------------------------------------------------

\end{document}

